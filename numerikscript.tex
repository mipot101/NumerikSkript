\documentclass[12pt]{article}


\usepackage{algorithmic} %Für Pseudocode https://math-linux.com/latex-26/faq/latex-faq/article/how-to-write-algorithm-and-pseudocode-in-latex-usepackage-algorithm-usepackage-algorithmic
\usepackage{stmaryrd} %Für Widerspruchsblitz
\usepackage{amsmath} 
\usepackage{amssymb}
\usepackage{amsthm} %Für Theoreme und Beweise
\usepackage{graphicx} %Für Bilder

\newtheoremstyle{break}% name
  {}%         Space above, empty = `usual value'
  {}%         Space below
  {\itshape}% Body font
  {}%         Indent amount (empty = no indent, \parindent = para indent)
  {\bfseries}% Thm head font
  {.}%        Punctuation after thm head
  {\newline}% Space after thm head: \newline = linebreak
  {}%         Thm head spec

\theoremstyle{break}

\renewcommand{\thesection}{\Roman{section}}
\renewcommand{\thesubsection}{\arabic{subsection}}

%Definiere Satz, Definition,...
\newtheorem{theorem}{Satz}[subsection]
\newtheorem{korollar}[theorem]{Korollar}
\newtheorem{definition}[theorem]{Definition}
\newtheorem{algorithm}[theorem]{Algorithmus}
\newtheorem{comment}[theorem]{Bemerkung}
\newtheorem{problem}[theorem]{Problem}
\newtheorem{example}[theorem]{Beispiel}
\newtheorem{nothing}[theorem]{}

\author{Prof. Schaedle}
\title{Numerik 1}

\begin{document}
\maketitle

\newpage

\section{Numerische Integration}

\subsection{Einführung}

\begin{problem}
Gegeben $f: [a,b] \rightarrow \mathbb{R}$ mit $a, b \in \mathbb{R}$.
Berechne $\int_a^b f(x) dx $
\end{problem}

\begin{example}\leavevmode
\begin{enumerate}
  \item Archimedes (282-212 v.Chr.): Fläche unter einer Parabel \\ \\
    \includegraphics[width=5cm]{Kapitel_1/Grafiken/Grafik_1.png} \\
    $A_{Parabel} = A_{Trapez} + \frac{4}{3} A_{Dreieck}$
  \item Leibniz + Newton (~1670):
    $$ \int_a^b f(x) dx = F(b) - F(a),$$ wobei $\frac{d}{dx} F(x) = f(x)$
  \item Riemann (~1850): 
    $$ \int_a^b f(x) dx = \lim\limits_{\vert \Delta \vert \to 0} \sum_{j=1}^n f(\xi_j)(x_j - x_{j-1}),$$
    wobei $\Delta = (x_0,...,x_n)$ Gitter Zerlegung von $[a, b]$, $a=x_0 < ...< x_n = b$, $\xi_j \in [x_{j-1}, x_j]$ und $\vert \Delta \vert := \max_{j=1,...n} \vert x_j - x_{j-1} \vert$.
    Das Riemannintegral existiert, falls: 
    $$ \forall \varepsilon > 0 \exists \delta > 0: \vert \Delta \vert < \delta \Rightarrow \vert \int_a^b f(X) dx - \sum_{j=1}^n f(\xi_j)(x_j-x_{j-1}) \vert < \varepsilon $$
\end{enumerate}
\end{example}

\begin{comment}[Approximation von Integralen]\leavevmode
\begin{enumerate}
  \item (linke) Rechtecksregel: 
    $$\int_{x_{j-1}}^{x_{j-1}+h} f(x) dx \approx h f(x_{j-1})$$
    $$\int_a^b f(x) dx = \sum_{j=1}^n \int_{x_{j-1}}^{x_j} f(x)dx \approx \sum_{j=1}^n f(x_{j-1}) (x_j-x_{j-1})$$
  \item Mittelpunktsregel:
    $$\int_{x_{j}}^{x_{j}+h} f(x) dx \approx f\left(\frac{x_j+x_j+h}{2}\right)h$$
    $$\int_a^bf(x)dx \approx \sum_{j=1}^n f\left( \frac{x_{j-1} + x_j}{2}\right) (x_j - x_{j-1})$$
    Da mit Hilfe der Transformationsformel sich jedes Integral $\int_{x_{j-1}}^{x_j}$ auf ein Integral $\int_a^b$ transformieren lässt, betrachten wir ohne Einschränkungen Integrale von $0$ bis $1$. Nutze dazu die Abb. $[a, b] \rightarrow [x_{j-1}, x_j], t \mapsto x_{j-1} + t(x_j - x_{j-1})$.
    $$ \int_{x_{j-1}}^{x_j} f(x) dx = \int_0^1 \underbrace{f\left( x_{j-1} + t(x_j - x_{j-1})\right)}_{:= g_{j-1}(t)} (x_j - x_{j-1})dt = \int_0^1 g_{j-1}(t)(x_j-x_{j-1})dt$$
\end{enumerate}
\end{comment}

\begin{definition}[Quadraturformel]
Eine s-stufige Quadraturformel zur Approximation von $\int_0^1 g(t)dt$ mit Knoten $c_i$ und Gewichten $b_i$ für $i=1,...s$ ist gegeben durch 
$$\sum_{i=1}^s b_i g(c_i) \left( \approx \int_0^1 g(t) dt\right)$$

\end{definition}

\begin{example}\leavevmode
\begin{enumerate}
  \item Rechtecksregel: $s = 1, b_1 = 1, c_1 = 0$
    $$ \int_0^1 g(t) \approx b_1 g(c_1) = g(0)$$
  \item Mittelpunktsregel: $s = 1, b_1 = 1, c_1 = \frac{1}{2}$
    $$ \int_0^1 g(t) \approx g\left(\frac{1}{2}\right)$$
  \item Trapezregel: $s=2, b_1=b_2= \frac{1}{2}, c_1 = 0, c_2 = 1$
    $$ \int_0^1 g(t) \approx \frac{1}{2} g(0) + \frac{1}{2}g(1)$$
  \item Simpsonregel: $s=3, b_1 =  \frac{1}{6}, b_2 =  \frac{2}{3}, b_3 =  \frac{1}{6}, c_1 = 0, c_2 =  \frac{1}{2}, c_3 = 1$
    $$ \int_0^1 g(t) \approx \frac{1}{6} \left(g(0) + 4g\left(\frac{1}{2}\right) +g(1)\right)$$ 
    \textbf{Herleitung:} Man legt eine Parabel $p$ durch die Punkte $(0, g(0)), (\frac{1}{2}, g(\frac{1}{2})), (1, g(1))$ und integriert $p$ von 0 bis 1. \\
    $p(t) = g(0)(1-t)2(\frac{1}{2}-t) + g(\frac{1}{2})(1-t)4t + g(1)(\frac{1}{2}-t)2t$ \\
    $$\Rightarrow \int_0^1 p(t)dt = \frac{1}{6}g(0)+ \frac{2}{3}g\left(\frac{1}{2}\right) +\frac{1}{6}g(1)$$ 
  \item "pulcherrima et utilissima regula" von Newton:
    $$\int_0^1 g(t) dt \approx \frac{1}{8}\left(g(0) + 3g\left(\frac{1}{3}\right) + 3g\left(\frac{2}{3}\right) + g(1)\right)$$
\end{enumerate}

\end{example}

\begin{comment}[Monte-Carlo Integration]\leavevmode
\begin{enumerate}
  \item Eindimensionale Monte-Carlo Integration: \\
    Sei $a, b \in \mathbb{R}$, $a<b$. Wählt man $N$ unabhängige gleichverteilte Punkte $x_i$ in $[a,b]$ so gilt die Approximation:
    $$\int_a^b f(x) dx \approx \frac{1}{N} \sum_{j=1}^N (b-a)f(x_j)$$
    Nach dem Gesetz der großen Zahlen konvergiert dieser Ausdruck, falls 
    $$\int_a^b\vert f(x) \vert dx < \infty, \int_a^b f^2(x)dx < \infty$$
  \item Mehrdimensionale Monte-Carlo Integration: \\
    Sei $W=\otimes_{i=1}^d [a_i, b_i]$ ein d-dimensionaler Quader. Wählt man in W unabh. gleichvert. Zufallsvektoren $x_i$ in W, so ist
    $$\int_W f(x)dx \approx \frac{1}{N} Vol(W) \sum_{i=1}^N f(x_i),$$
    wobei $f:\mathbb{R}^d \rightarrow \mathbb{R}$.\\
    \textbf{Achtung:} Dieses gewöhnliche MC-Verfahren konvergiert sehr langsam. Verbesserungen sind z.B.: Importance sampling, Control variates, Antithetic variates und statified sampling.
\end{enumerate}
\end{comment} %Einführung
\subsection{Ordnung von Quadraturformeln}

\begin{definition}
Eine Quadraturformel (QF) mit Gewichten und Knoten $(b_i, c_i)_{i=1}^{s}$ hat \textbf{Ordnung p}, falls sie exakt ist für alle Polynome von Grad $ \leq p-1$.\\
$\mathcal{P}$: Menge aller Polynome $$\left\{ \sum_{i=0}^{n}a_i*X^i , a_i \in \mathbb{R} (\mathbb{C}) \right\}  $$ \\
deg(q): Grad des Polynoms
\end{definition}

\begin{theorem}
Ein QF $(b_i, c_i)_{i=1}^{s}$ für $[0,1]$ hat Ordnung $p$ genau dann, wenn
$$\sum_{i=1}^{s} b_i c_i^{q-1} = \frac{1}{q}$$ für $q = 1,..,p$
\begin{proof}[Beweis]
\begin{description}
  \item
\end{description}
\begin{description}
  \item $"\Rightarrow"$ \\
  QF hat Ordnung p $\Rightarrow$ QF ist exakt für $g(t) = t^{q-1}$ für $q = 1,..,p$ auf $[0,1]$ \\
  $\Rightarrow$ $$\sum b_i c_i^{q-1} = \int_{0}^{1} t^{q-1} dt = [\frac{t^q}{q}]_{t=0}^{1} = \frac{1}{q}$$
  \item $"\Leftarrow"$\\
  Jedes Polynom von Grad $p-1$ lässt sich als Linearkombination von $1, t, t^2, ...,t^{p-1}$. Die Behauptung folgt aus der Linearität in $g$ von $$\int_{0}^{1} g(t) dt$$ und $$\sum_{i=1}^{s}b_i g(c_i)$$
\end{description}
\end{proof}
\end{theorem}

\begin{example}
\begin{description}
  \item
\end{description}

\begin{enumerate}
  \item Rechtecksregel: $p=1$
  \item Mittelpunktsregel: $p=2$
  \item Trapezregel: $p=2$
  \item Simpsonregel: $p \geq 3$ nach Konstruktion \\
  $q = 4$: $1/6 * 0^3 + 4/6 * (1/2)^3 + 1/6 * 1^3 = 1/4 = 1/4$ \\
  $q = 5$: $1/6 * 0^4 + 4/6 * (1/2)^4 + 1/6 * 1^4 = 5/24 \neq 1/5$ \\
  Damit ist die Ordnung 4!
  \item "pulcherina et utilissima": Übung
\end{enumerate}
\end{example}

\begin{comment}
Zu vergebenen paarweise verschiedenen Knoten $c_1, ..., c_s$ lässt sich aus (*) für $p=s$ ein lineares Gleichungssystem für die Gewichte $b_1, ..., b_s$ aufstellen.\\

$$
\underbrace{\left[ \begin{array}{rrrr}
1 & 1 & ... & 1 \\
c_1 & c_2 & ... & c_s \\
... & ... & ... & ... \\
c_1^{s-1} & c_2^{s-1} & ... & c_s^{s-1} \\
\end{array}\right]}_{= V}
*
\left[ \begin{array}{r}
b_1 \\
b_2 \\
... \\
b_s \\
\end{array}\right] 
= 
\left[ \begin{array}{r}
1 \\
1/2 \\
... \\
1/s \\
\end{array}\right] 
$$
Falls die Vandermonde-Matrix V invertierbar ist, so lassen sich die Gewichte $b_1, ..., b_s$ bestimmen, sodass die QF $(b_i, c_i)_{i = 1}^{s}$ mindestens Ordnung $s$ hat.
\end{comment}

\begin{definition}
Eine QF heißt symmetrisch, falls für $i = 1,...,s$
\begin{enumerate}
  \item $c_i = 1 - c_{s+1-i}$
  \item $b_i = b_{s+1-i}$
\end{enumerate}
\end{definition}

\begin{example}
MP, TP, Simpson,...
\end{example}

\begin{theorem}
Die maximal erreichbare Ordnung einer symmetrischen QF ist gerade.
\begin{proof}[Beweis]
Sei die QF $(b_i,c_i)_{i=1}^{s}$ exakt for Polynome vom Grad $\leq 2m-2$ (für $m \in \mathbb{N}$), (dann ist die Ordnung $\geq 2m-1$).
$$\forall g \in \mathcal{P}: deg(g) \leq 2m-2 \Rightarrow \sum_{i=1}^{s} b_i g(c_i) = \int_{0}^{1} g(t) dt$$
Sei $f \in \mathcal{P}$ mit $deg(f) = 2m-1$. \\
Wir zeigen QF ist exakt für $f$. 
$$f(t) = ct^{2m-1} + g(t)$$
für $g \in \mathcal{P}$ mit $deg(g) \leq 2m-2$ mit $c \neq 0$. \\
Trick: $f(t) = c(t-\frac{1}{2})^{2m-1} + \tilde{g}(t)$ mit $\tilde{g} \in \mathcal{P}$ und $deg(\tilde{g}) \leq 2m-2$

\begin{enumerate}
  \item Für $\tilde{g}$ ist die QF exakt
  \item $$\int_0^1 (t-\frac{1}{2})^{2m-1} dt = \left[\frac{1}{2m-2}(t-\frac{1}{2})^{2m-2}\right]_0^1 = 0$$
  $$ \sum_{i=1}^{s} b_i (c_i - \frac{1}{2})^{2m-1}$$ 
  Symmetrie $\Rightarrow$
  $$= \sum_{i=1}^{s} b_{s+1-i} (\frac{1}{2} - c_{s+1-i})^{2m-1} $$
  Definiere $j := s+1-i$
  $$ = \sum_{i=1}^{s} b_i \frac{1}{2} - c_i)^{2m-1} = -\sum_{i=1}^{s} b_i (c_i - \frac{1}{2})^{2m-1}$$
  $$\Rightarrow 2*\sum_{i=1}^{s} b_i (c_i - \frac{1}{2})^{2m-1} = 0$$
  $$\Rightarrow \sum_{i=1}^{s} b_i (c_i - \frac{1}{2})^{2m-1} = 0$$
  $$\sum_{i=1}^{s}b_if(c_i) = c \sum_{i=1}^{s}b_i(c_i-\frac{1}{2})^{2m-1} + \sum_{i=1}^{s}b_i\tilde{g}(c_i)$$
  $$ = c\int_0^1(t-\frac{1}{2})^{2m-1} dt + \int_0^1 \tilde{g}(t)dt = \int_0^1 f(t)dt$$
  $\Rightarrow$ QF hat mind. Ordnung $2m$.
\end{enumerate}
\end{proof}
\end{theorem}

\begin{theorem}
Sind Knoten $c_1 < c_2 < ... < c_s$ ($c_i \in \mathbb{R}, i = 1,...s$) gegeben, so existieren eindeutig bestimmte Gewichte $b_1 ,..., b_s$ derart, dass die QF $(b_i, c_i)_{i=1}^s$ die maximale Ordnung $p \geq s$ hat. \\
Es gilt $$b_i = \int_0^1 l_i(t) dt$$ mit $$l_i(t) = \frac{\prod_{j=1, j\neq i}^s (t-c_j)}{\prod_{j=1, j\neq i}^s (c_i-c_j)}$$
Bemerkung/Definition
\begin{description}
  \item $l_i$ ist das i-te Lagrangepolynom zu den Knoten $c_i, ...,c_s$. Es gilt $deg(l_i) = s-1$ 
  $$l_i(c_j) = \left\{
\begin{array}{ll}
0 & \,i \neq j \\
1 & \, i = j\\
\end{array}
\right. $$
\end{description}

\begin{proof}[Beweis] von 2.8 \\
\begin{enumerate}
  \item Hat die QF die Ordnung $p \geq s$, so ist wegen $deg(l_i) = s-1)$:
  $$ \int_0^1 l_i(t) dt = \sum_{j=1}^s b_j l_i(c_j) = b_i$$
  \item Zu den Knoten $c_i, ...c_s$ definiere $b_i$ wie angegeben. Die QF ist dann exakt für alle Polynome von Grad $ \leq s-1$, da die $l_1, ...,l_s$ linear unabhängig sind und eine Basis des Vektorraums der Polynome von Grad $\leq s-1$ bilden.
\end{enumerate}
\end{proof}
\end{theorem}
 %Ordnung von Quadraturformeln
\subsection{Quadraturfehler}

\underline{Allgemeine Voraussetzung:} $f:[a,b] \rightarrow \mathbb{R}$ sei hinreichend oft differenzierbar ($f$ ist eine glatte Funktion)

\begin{definition}
Der Fehler bei der Approximation des Integrals durch die QF ist 
\begin{align*}
err &= \int_a^b f(x)dx - \sum_{j=0}^{n-1}\left( h_{j+1} \sum_{i=1}^s b_i f(x_j+h_{j+1}c_i)\right)
\intertext{mit $h_{j+1} = x_{j+1}-x_j$}
&= \sum_{j=0}^{n-1} \left( \int_{x_j}^{x_{j+1}} f(x_j + \tau) d\tau - h_{j+1} \sum_{i=1}^s b_i f(x_j + h_{j+1}c_i)\right) &\\
&= \sum_{j=0}^{n-1} h_{j+1} \int_0^1 g_j(\xi) d\xi -h_{j+1} \sum_{i=1}^s b_i g_j(c_i)
\intertext{mit $ g_j(\xi) = f(x_j + \xi h_{j+1})$}
\end{align*}
Der Quadraturfehler auf Teilintervallen $[x_j, x_j+h_{j+1}]$ ist 
\begin{align*}
E(f, x_j, h_{j+1}) &= \int_{x_j}^{x_{j+1}} f(x)dx  - h_{j+1} \sum_{i=1}^s b_i f(x_j + c_i h_{j+1}) &\\
&= h_{j+1} \left( \int_0^1 g_j(\xi) d\xi - \sum_{i=1}^s b_i g_j(c_i) \right)
\end{align*}
\end{definition}

\begin{nothing}[Fehlerabschätzung - 1. Versuch]
Falls $f$ auf $[x_0, x_0+h]$ glatt genug ist und die QF Ordnung $p$ hat, aber nicht Ordnung $p+1$, so erhält man durch Taylorentwicklung um $x_0$ von $f(x_0 + \xi h) = g_0(\xi)$ und $f(x_0+c_ih)$:
\begin{align*}
E(f, x_0, h) &= \sum_{k\geq 0} \frac{h^{k+1}}{k!} \left( \int_0^1 t^k dt - \sum_{i=1}^s b_i c_i^k \right) f^{(k)}(x_0)&\\
&= \frac{h^{p+1}}{p!} \left( \frac{1}{p+1} - \sum_{i=1}^s b_i c_i^p\right) f^{(p)}(x_0) + \underbrace{\mathcal{O}(h^{p+2})}_{Taylorrestglied}
\end{align*}
Die Konstante $C = \frac{1}{p!} \left( \frac{1}{p+1} - \sum_{i=1}^s b_i c_i^p \right)$ heißt Fehlerkonstante. \\
Ist $h$ klein genug, sodass das Taylorrestglied im Vergleich zu $h^{p+1}C f^{(p)}(x_0)$ vernachlässigbar ist, so gilt:
\begin{align*}
err &= \sum_{j=0}^{n-1} E(f, x_j, h)
\intertext{mit $ x_j = x_0+jh$}
&\approx Ch^p \sum_{j=0}^{n-1} hf^{(p)}(x_j)&\\
&\approx Ch^p \int_a^b f^{(p)}(x)dx&\\
&= Ch^p \left(f^{(p-1)}(b) - f^{(p-1)}(a) \right)
\end{align*}
\end{nothing}

\begin{nothing}[Rigorose Fehlerabschätzung]
\begin{description}
  \item
\end{description}
\begin{description}
  \item[Satz 1:]
    Sei $f: [a, b] \rightarrow \mathbb{R}$ $k$-mal stetig differenzierbar ($f \in C^k([a, b])$) und habe die QF Ordnung $p$, so gilt für $h < b-a$ und $k \leq p$\\
    $$E(f, x_0, h) = h^{k+1} \int_0^1 K_k(\tau) f^{(k)}(x_0+\tau k) d\tau,$$
    wobei der Peanokern $K_k(\tau)$ durch 
    $$ K_k(\tau) := \frac{(1-\tau)^k}{k!} - \sum_{i=1}^s b_i \frac{(c_i - \tau)^{k-1}_+}{(k-1)!}, $$
    mit 
    $(\sigma)_+^{k-1} = \left\{
    \begin{array}{ll}
    \sigma ^{k-1} &  \sigma > 0 \\
    0 & \, \textrm{sonst} \\
    \end{array}
    \right. $, gegeben ist.
  \item \begin{proof}[Beweis] 
    Taylorentwicklung mit Integralrestglied und Transformation 
    $$f(x_0 + th) = \sum_{j=0}^{k-1} \frac{(th)^j}{j!} f^{(j)}(x_0) + h^k \int_0^t \frac{(t-\tau)^{k-1}}{(k-1)!} f^{(k)}(x_0+\tau h) d\tau$$
    eingesetzt in (*) und die Verwendung von 
    $$\int_0^{c_i} (c_i - \tau)^{k-1} g(\tau) d\tau = \int_0^1 (c_i-\tau)_+^{k-1} g(\tau) d\tau$$
    liefern
    \begin{align*} 
    E(f, x_0, h) &= h \int_0^1 \left( \sum_{j=0}^{k-1} \frac{(th)^j}{j!} f^{(j)}(x_0) + h^k \int_0^t \frac{(t-\tau)^{k-1}}{(k-1)!} f^{(k)}(x_0+\tau h) d\tau \right)dt &\\
    &\quad- h \sum_{i=1}^s b_i \left( \sum_{j=0}^{k-1} \frac{(c_ih)^j}{j!} f^{(j)}(x_0) + h^k \int_0^{c_i} \frac{(c_i-\tau)^{k-1}}{(k-1)!} f^{(k)}(x_0+c_ih) d\tau \right) &\\
    &\underset{k\leq p}{=} h h^k \left( \int_0^1 \int_0^t \frac{(t-\tau)^{k-1}}{(k-1)!} f^{(k)} (x_0 + \tau h) d\tau dt \right) &\\
    &\quad- h h^k \left(\sum_{i=1}^s \int_0^1 \frac{(c_i-\tau)^{k-1}_+}{(k-1)!} f^{(k)}(x_0+\tau h) d\tau \right)&\\
    &= h h^k \left( \int_0^1 \int_0^1 \frac{(t-\tau)^{k-1}_+}{(k-1)!} f^{(k)}(x_0+\tau h) d\tau dt \right) &\\
    &\quad - h h^k \left(\sum_{i=1}^s b_i \int_0^1 \frac{(c_i - \tau)^{k-1}_+}{(k-1)!} f^{(k)}(x_0+\tau h) d\tau \right)&\\
    &= h^{k+1} \int_0^1 \left( \int_0^1 \frac{(t-\tau)_+^{k-1}}{(k-1)!} dt - \frac{(c_i - \tau)_+^{k-1}}{(k-1)!} \right) f^{(k)}(x_0+\tau h) d\tau &\\
    &= h^{k+1} \int_0^1 K_k(\tau) f^{(k)}(x_0+\tau h) d\tau, &\\
    \end{align*}
    
    da
    \begin{displaymath}
    \int_0^1 \frac{(t-\tau)^{k-1}_+}{(k-1)!} dt
    = \int_0^1 \frac{(t-\tau)^{k-1}}{(k-1)!}
    = \left[ \frac{1}{k!} (t-\tau)^k \right]_{t=\tau}^1
    = \frac{1}{k!} (1-\tau)^k
    \end{displaymath}
    gilt.
    \end{proof}
    
  \item[Satz 2:] (Eigenschaften des Peanokerns) \\
    Für eine QF der Ordnung $p$ gilt für $k \leq p$ ($k, p \in \mathbb{N}$) 
    \begin{enumerate}
      \item $K_k'(\tau) = -K_{k-1}(\tau)$ für $k \geq 2$ und $\tau \neq c_i$ falls $k=2$
      \item $K_k(1) = 0$ für $k \geq 1$, falls $c_i \leq 1$ für $i=1,..., s$
      \item $K_k(0) = 0$ für $k \geq 2$, falls $c_i \leq 1$ für $i=1,..., s$
      \item $\int_0^1 K_p(\tau) = \frac{1}{p!} \left(\frac{1}{p-1} - \sum_{i=1}^s b_i c_i^p \right)=: C$ (Fehlerkonstante $C$ aus (3.2))
      \item $K_1(\tau)$ ist stückweise linear mit Steigung $-1$ und Sprüngen der Höhe $b_i$ an den Stellen $c_i$
    \end{enumerate}
  
  \item
  \begin{proof}[Beweis]
    Eventuell Übungsaufgabe
  \end{proof}
  
  \item[Beispiel:] 
    Mittelpunktsregel: 
      \begin{align*}
      K_1(\tau) &= \frac{(1-\tau)^1}{1!} - 1 \frac{(\frac{1}{2} - \tau)^1_+}{0!} &\\
      &= 1- \tau - \left( \frac{1}{2} - \tau \right)_+^0&\\
      &= \left\{
        \begin{array}{ll}
        1-\tau - 1  & \tau < \frac{1}{2} \\
        1-\tau & \, \tau \geq \frac{1}{2} \\
        \end{array}
      \right.
      \end{align*}
      
      \begin{align*}
      K_2(\tau) &= \frac{(1-\tau)^2}{2!} - 1 \frac{(\frac{1}{2} - \tau)^1_+}{1!}&\\
      &= \frac{1}{2} (1-\tau)^2 - \left( \frac{1}{2} - \tau \right)_+^1&\\
      &= \left\{
        \begin{array}{ll}
        \frac{\tau^2}{2}  & \tau < \frac{1}{2} \\
        \frac{1}{2}(1-\tau)^2 & \, \tau \geq \frac{1}{2} \\
        \end{array}
      \right.
      \end{align*}
    
    \item[Satz 3:] 
      Sei $f \in C^k([a,b])$ und habe die QF $(b_i, c_i)^s_{i=1}$, Ordnung $p \geq k$, so gilt für den Fehler $err$ aus $(3.1)$ 
      $$ \vert err \vert \leq h^k (b-a) \int_0^1 \vert K_k(\tau) \vert d\tau \max_{x \in [a,b]} \vert f^{(k)}(x) \vert, \quad \quad h = \max_{j=1,..,n} h_j.$$
    
    \item \begin{proof}[Beweis]
      Mit Satz $1$ gilt 
      \begin{align*}
      \vert E(f, x_j, h_{j+1}) \vert &\leq h_{j+1}^{k+1} \int_0^1 \vert K_k(\tau) \vert \vert f^{(k)}(x_j+\tau h_{j+1}) \vert d\tau&\\
      &\leq h_{j+1}^{k+1} \int_0^1 \vert K_k(\tau) \vert d\tau \max_{x \in [x_j, x_j + h_{j+1}]} \vert f^{(k)}(x) \vert&\\
      \end{align*}
      Zudem gilt 
      \begin{align*} 
      \vert err \vert &= \left\vert \sum_{j=0}^{n-1} E(f, x_j, h_{j+1}) \right\vert &\\
      &\leq \sum_{j=0}^{n-1} \vert E(f, x_j, h_{j+1}) \vert &\\
      &\leq \underbrace{\sum_{j=0}^{n-1} h_{j+1}}_{b-a} \underbrace{h_{j+1}^k}_{\leq h^k} \int_0^1 \vert K_k(\tau) \vert d\tau \underbrace{\max_{x \in [x_j, x_{j+1}]} \vert f^{(k)} (x) \vert}_{\leq \max_{x\in[a,b] \vert f^{(k)}(x) \vert}}
      \end{align*}
      Damit folgt die Behauptung.
    \end{proof}
    
    \item[Beispiele]
      \begin{description}\item\end{description}
      \begin{description}
        \item Für die Mittelpunktsregel (maximale Ordnung = 2) erhält man
      		$$ \vert err \vert \leq h^2 (b-a) \frac{1}{24} \max_{x\in[a,b]} \vert f^{(2)}(x) \vert $$ 
      	\item Für die Trapezregel (maximale Ordnung = 2)
      		$$ \vert err \vert \leq h^2 (b-a) \frac{1}{12} \max_{x\in[a,b]} \vert f^{(2)}(x) \vert $$  
      	\item Für die Simpsonregel (maximale Ordnung = 4)
      		$$ \vert err \vert \leq h^4 (b-a) \frac{1}{2880} \max_{x\in[a,b]} \vert f^{(4)}(x) \vert $$ 
	  \end{description}
      $\rightarrow$ Der Fehler wird klein, falls $h$ klein und die Ordnung $p$ groß wird.
\end{description}
\end{nothing} %Quadraturfehler
\subsection{Quadratur mit hoher Ordnung}
Es seien Knoten $c_1< ... < c_s$ gegeben. Aus $\S2$ wissen wir: \\
Es gibt Gewichte $b_1, ..., b_s$, sodass $p \geq s$. \\
\underline{Fragen:} 
\begin{itemize}
  \item Kann man $c_j$ so wählen, dass $p>s$?
  \item Wenn ja, wie?
  \item Wie groß kann $p$ maximal werden?
\end{itemize}
\underline{Ziel:} QF mit Ordnung $p=s+m$ für $m \in \mathbb{N}, m > 1$.
Sei $g \in \mathcal{P}_{s+m-1}$ (Polynome von Grad $\leq s+m-1$).\\
$g$ soll durch die QF exakt integriert werden.\\
\underline{Idee:} Dividiere $g$ durch $M(t) = \prod_{i=1}^s (t-c_i)$ "Knotenpolynom"\\
$deg(M) = s$ \\
$g(t) = M(t) h(t) + r(t)$ mit Rest $r$, $deg(r) \leq s-1$ und $deg(h) \leq m-1$ \\
Dann gilt einerseits
$$\int_0^1 g(t)dt = \int_0^1 M(t)h(t)dt + \int_0^1r(t)dt$$
und andererseits
\begin{align*}\sum_{i=1}^s b_ig(c_i) &= \sum_{i=1}^s b_i \underbrace{M(c_i)}_{= 0} h(c_i) + \sum_{i=1}^s b_ir(c_i) &\\
 &= 0 + \int_0^1 r(t)dt,\end{align*}
 da $p \leq s$\\
Damit ist gezeigt:

\begin{theorem}
Sei $(b_i, c_i)_{i=1}^s$ der Ordnung $p \geq s$. Äquivalent sind:
\begin{enumerate}
  \item QF hat Ordnung $s+m$
  \item $\forall h \in \mathcal{P}_{m-1}:\int_0^1 M(t)h(t)dt = 0$
\end{enumerate}
\end{theorem}

\begin{korollar}
Die Ordnung einer $s$-stufigen QF ist höchstens $2s$.
\begin{proof}[Beweis (indirekt)]
Annahme: $p > 2s$ \\
$$(4.1) \Rightarrow \forall h \in \mathcal{P}_s: \int_0^1 M(t)h(t)dt = 0 $$
Setze $h=M$, dann ist $$ \int_0^1 M(t)^2dt = 0$$
$\lightning$ zu $\int_0^1 M(t)^2 dt > 0$, da $M(t) \equiv 0$
\end{proof}
\end{korollar}

\begin{nothing}[Beispiele/Korollare]
\begin{description} \item \end{description}
\begin{enumerate}
  \item Jede 3-stufige QF mit Ordnung $\geq 4$ muss
    \begin{align*}
    &\int_0^1 (t-c_1)(t-c_2)(t-c_3)dt = 0 &\\
    \Leftrightarrow &\int_0^1 t^3 + t^2(-c_1-c_2-c_3) + t(c_1c_2+c_2c_3+c_1c_3) - c_1c_2c_3 dt &\\
    &= \frac{1}{4} + \frac{1}{3}(-c_1-c_2-c_3) + \frac{1}{2}(c_1c_2 + c_2c_3 + c_1c_3) - c_1c_2c_3 = 0
    \intertext{erfüllen, dh.}
    c_3 &= \frac{\frac{1}{4} - (c_1+c_2)\frac{1}{3} + c_1c_2 \frac{1}{2}}{\frac{1}{3} - (c_2+c_1)\frac{1}{2} + c_1c_2}
    \end{align*}
    
  \item Zur Berechnung der Knoten einer $3$-stufigen QF der Ordnung $6$ verwenden wir $(4.2)$ mit $h(t) = 1, t, t^2$
    $$\int_0^1 M(t)h(t) = 0$$
    \begin{flalign*}
    &h(t) = 1 &\Rightarrow \quad &c_1c_2c_3 - \frac{1}{2}(c_1c_2 + c_2c_3 + c_1c_3) + \frac{1}{3}(c_1+c_2+c_3) = \frac{1}{4} &\\
    &h(t) = t &\Rightarrow \quad &\frac{1}{2}c_1c_2c_3 - \frac{1}{3}(c_1c_2 + c_2c_3 + c_1c_3) + \frac{1}{4}(c_1+c_2+c_3) = \frac{1}{5}&\\
    &h(t) = t^2 &\Rightarrow \quad &\frac{1}{3}c_1c_2c_3 - \frac{1}{4}(c_1c_2 + c_2c_3 + c_1c_3) + \frac{1}{5}(c_1+c_2+c_3) = \frac{1}{6}
    \end{flalign*}
    Das ist ein nichtlineares Gleichungssystem in $c_1, c_2, c_3$. \\
    \underline{Trick:}
    \begin{description}
      \item $\sigma_1 = c_1+c_2+c_3$
      \item $\sigma_2 = c_1c_2 + c_1c_3 + c_2c_3$
      \item $\sigma_3 = c_1c_2c_3$
    \end{description} 
    Das sind die Koeffizienten von $M(t)$ in der Monombasis. \\
    $M(t) = (t-c_1)(t-c_2)(t-c_3) = t^3 - \sigma_1t^2 + \sigma_2t - \sigma_3$ \\
    und das Gleichungssystem ist linear in $\sigma_1, \sigma_2, \sigma_3$ \\
    mit Lösung $\sigma_1 = \frac{3}{2}, \sigma_2 = \frac{3}{5}, \sigma_3 = \frac{1}{20}$ \\
    und damit ist 
    \begin{align*}
    M(t) &= t^3 - \frac{3}{2}t^2 + \frac{3}{5}t - \frac{1}{20}&\\
    &= (t-\frac{1}{2})(t-\frac{5-\sqrt{15}}{10})(t-\frac{5 + \sqrt{15}}{10}) &\\
    \end{align*}
    Glücklicherweise sind die Wurzeln von $M(t)$ in $[0,1]$. Damit lassen sich die Gewichte mit $(2.4)$ berechnen und wir erhalten
    $$\int_0^1 g(t) dt = \frac{5}{18} g\left(\frac{5-\sqrt{15}}{10}\right) + \frac{8}{18} g\left(\frac{1}{2}\right) + \frac{5}{18}g\left(\frac{5+\sqrt{15}}{10}\right)$$
\end{enumerate}
\end{nothing}
\underline{Ziel:} Konstruktion von QF der Ordnung $2s$ mit Hilfe von orthogonalen Polynomen.
 %Quadratur mit hoher Ordnung
\subsection{Orthogonalpolynome}
Bedingung $2.$ in Satz $(4.1)$ 
$$ \forall h \in \mathcal{P}_{m-1}: \int_0^1 M(t)h(t) = 0$$
kann als Orthogonalitätsbedingung bzgl. eines Skalarprodukts $\langle f, g\rangle = \int_0^1 f(t)g(t)dt$ auf dem Vektorraum $L^2([0,1])$ oder $C([0,1])$ aufgefasst werden. \\
\underline{Erinnerung:}
$$\mathcal{P}_s := \left\{ \sum_{j=0}^s \alpha_j X^j, \alpha_j \in \mathbb{R} \right\}$$ 
ist ein $\mathbb{R}$-VR mit $dim(\mathcal{P}_s) = s+1$ und Basis $\left\{ 1, X, X^2, ..., X^s \right\}$\\ \\
$\langle\cdot,\cdot\rangle : C([0,1]) \times C([0,1]) \rightarrow \mathbb{R}, (f, g) \mapsto \int_0^1 f(t)g(t)dt$ ist 
\begin{enumerate}
  \item symmetrisch $ \langle f, g\rangle = \langle g, f\rangle$
  \item linear $\langle \alpha f + g, h\rangle = \alpha \langle f, h\rangle + \langle g, h\rangle$
  \item positiv definit $\langle f, f\rangle \geq 0 $ und $ \langle f, f\rangle = 0 \Rightarrow f = 0$
\end{enumerate}
Wie in der linearen Algebra definieren wir $f$ steht senkrecht auf $g$: $f \perp g \Leftrightarrow \langle f, g\rangle = 0$

\begin{theorem}
QF hat die Ordnung $s+m \Leftrightarrow $ M ist orthogonal auf allen Polynome in $\mathcal{P}_{m-1}$
\end{theorem}

\begin{definition}
Für eine Gewichtsfunktion $\omega : (a, b) \rightarrow \mathbb{R}$ mit 
\begin{enumerate}
  \item $\omega$ stetig
  \item $\forall x\in(a,b): \omega(x) > 0 $
  \item $\forall k \in \mathbb{N}: \int_a^b \omega(x) \vert x \vert^k dx < \infty$
\end{enumerate}
definieren wir auf den Vektorraum 
$$ V = \left\{ f: [a,b] \rightarrow \mathbb{R}: f \medspace stetig \medspace und \int_a^b f(x)^2 \omega(x) dx < \infty \right\} $$
das gewichtete Skalarprodukt
$$ \langle f, g \rangle_\omega := \int_a^b \omega(x)f(x)g(x)dx$$
für $f, g \in V$. \\
Zudem definiere:
$$f \perp_\omega g :\Leftrightarrow \langle f, g, \rangle_\omega = 0$$
\end{definition}

\begin{theorem}
Es existiert eine eindeutige Folge von Polynomen $p_0, p_1, ...$ mit 
\begin{enumerate}
  \item $deg(p_k) = k$
  \item $\forall q \in \mathcal{P}_{k-1}:p_k \perp q$ für $k \geq 1$
  \item $p_k(x) = x^k + r$ mit $deg(r) \leq  k-1$ "Normierung"
\end{enumerate}
Diese Polynome lassen sich rekursiv berechnen durch
\begin{flalign*}
&p_0(x) := 1\\
&p_{1}(x) := x \\
&p_{k+1}(x) := (x- \beta_{k+1}) p_k(x) - \gamma_{k+1}^2 p_{k-1}(x), \quad \text{für $k \geq 2$} \\
\intertext{Wobei $\beta$ und $\gamma$ definiert sind durch:}
&\beta_{k+1} := \frac{\langle xp_k, p_k \rangle}{\langle p_k, p_k \rangle} \\
&\gamma_{k+1}^2 := \frac{\langle p_k, p_k \rangle}{\langle p_{k-1}, p_{k-1} \rangle}
\end{flalign*}

\begin{proof}[Beweis]
(vgl. Gram-Schmidt Orthogonalisierung LinA) \\
Sei $p_0, ..., p_k$ bereits bekannt. Zur Konstruktion von $p_{k+1}$ setzen wir
$$p_{k+1}(x) = xp_k(x) + \sum_{j=0}^{k} \alpha_j p_j(x)$$
(damit ist 3. erfüllt) \\
Zur Bestimmung der $\alpha_j$:
\begin{enumerate}
  \item $ 0 = \langle p_{k+1}, p_k \rangle = \langle xp_k, p_k \rangle + \alpha_k \langle p_k, p_k \rangle + \sum_{j=0}^{k-1} \alpha_j \underbrace{\langle p_j, p_k \rangle}_{= 0}$\\
    $\Rightarrow \alpha_k = -\frac{\langle xp_k, p_k \rangle}{\langle p_k, p_k \rangle} =: -\beta_{k+1}$
  
  \item $$0 = \langle p_{k+1}, p_{k-1} \rangle = \langle xp_k, p_{k-1} \rangle + 0 + \alpha_{k-1} \langle p_{k-1}, p_{k-1} \rangle + 0$$ 
    $$= \langle p_{k}, xp_{k-1} \rangle + \alpha_{k-1} \langle p_{k-1}, p_{k-1} \rangle$$ 
    Aufgrund von 3. $\Rightarrow$ 
    $$xp_{k-1} = p_k + r$$ mit $deg(r) \leq k-1$
    $$\Rightarrow \langle p_{k}, xp_{k-1} \rangle = \langle p_{k}, p_{k} \rangle + \underbrace{\langle p_{k}, r \rangle}_{= 0}$$ 
    $$\Rightarrow \alpha_{k-1} = - \frac{\langle p_k, p_k \rangle}{\langle p_{k-1}, p_{k-1} \rangle} =: -\gamma_{k+1}^2$$
  
  \item Für $j \leq k-2$:
    $$ 0 = \langle p_{k+1}, p_{j} \rangle = \langle xp_{k}, p_{j} \rangle + \alpha_j \langle p_{j}, p_{j} \rangle$$
    $$ = \underbrace{\langle p_{k}, xp_{j} \rangle}_{= 0} + \alpha_j \underbrace{\langle p_{j}, p_{j} \rangle}_{\neq 0}$$
    $\langle p_{k}, xp_{j} \rangle = 0$ gilt, da $deg(xp_j) \leq k+1$ \\
    Insgesamt haben wir
    $$p_{k+1}(x) = xp_k(x) - \beta_{k+1}p_k(x) - \gamma_{k+1}^2 p_{k-1}(x)$$
\end{enumerate}
\end{proof}
\end{theorem}

\begin{comment*}
Für eine QF maximaler Ordnung müssen nach Satz (4.1) die Knoten $c_i$, $i=1, ...,s$ so gewählt werden, dass 
$$M(t) = \prod_{i=1}^s(t-c_i)$$
das Orthogonalpolynom vom Grad $s$ bezüglich des Skalarprodukts mit $\omega(x) \equiv 1$ auf $[0,1]$ ist. \\
    \underline{Frage:} Sind die Wurzeln (Nullstellen) der Orthogonalpolynome aus (5.3) reell? (Spoiler: Ja)
\end{comment*}

\begin{theorem}
Sei $p_k$ das Orthogonalpolynom wie in (5.3) definiert (bzgl. $\langle f, g \rangle = \int_a^b f(x)g(x)\omega(x)dx$). Alle Wurzeln von $p_k$ sind einfach und liegen im offenen Intervall $(a,b)$.

\begin{proof}[Beweis]
Seie $x_1, ..., x_r$ jene Wurzeln in $p_k$, die reell sind, in $(a, b)$ liegen und bei denen $p_k$ das Vorzeichen wechselt (Wurzeln mit ungerader Vielfachheit). \\
Klar ist: $r \leq k$. \\
Sei 
$$g(x) = \prod_{j=1}^r (x-x_j)$$ 
Dann ist 
$$ \langle p_k, g \rangle = \int_a^b \underbrace{p_k(x) \thinspace g(x)}_{\text{Wechselt das Vorzeichen in (a,b) nicht}} \thinspace \omega(x) \thinspace dx \neq 0$$
Andererseits ist $p_k$ orthogonal zu allen Polynomen vom Grad $\leq k-1$ \\
$\Rightarrow r = deg(g) \geq k$ \\
$\Rightarrow r=k$
\end{proof}
\end{theorem}

\begin{example}[Orthogonale Polynome]
\begin{tabular}{llll}
 
Bezeichnung & $(a,b)$ & $\omega(x)$ & Name\\
 
& & & \\

$P_k$ & $(-1,1)$ & $1$ & Legendrepolynome \\

$T_k$ & $(-1,1)$ & $(1-x^2)^{-1/2}$ & Tschebyscheff-Polynome \\

$P_k^{(\alpha, \beta)}$ & $(-1,1)$ & $(1-x)^{\alpha}(1-x)^{\beta}$ & Jacobi-Polynome $\alpha, \beta > -1$ \\

$L_k^{(\alpha)}$ & $(0, \infty)$ & $x^{\alpha} e^{-x}$ & Laguere-Polynome \\

$M_k$ & $(-\infty,\infty)$ & $e^{-x^2}$ & Harmitepolynome \\

\end{tabular}\\
\underline{Bemerkung:} Teilweise sind andere Normierungen üblich $P_k(1) = 1$, $T_k(x) = 2^{k-1} x^k + ...$, ...
\end{example}
 %Orthogonalpolynome
\subsection{Ein adaptives Programm}

Gegeben sei eine QF mit $(b_i, c_i)_{i=1}^s$ mit Ordnung $p=2s$ (die höchste Ordnung, die es gibt) z.B. $s=15$ \\
\underline{Ziel:} Ein Computerprogramm adagaussqf(f, a, b, Tol), welches für eine Funktion $f$ auf dem Interval $[a, b]$ eine Approximation an $\int_a^b f(x) dx$ berechnet, sodass der Fehler $\leq$ Tol ist (für viele Funktionen). \\
Konstruiere eine Zerlegung $\Delta = \left\{ a = x_0 < ... < x_n = b\right\}$ des Intervalls, sodass für die Approximation 
$$I_\Delta := \sum_{j=0}^{n-1} h_{j+1} \sum_{i=1}^s b_if(x_j + c_ih_{j+1})$$
gilt 
$$\left\vert I_\Delta - \int_a^b f(x) dx \right\vert \leq Tol \int_a^b \vert f(x) \vert dx $$
Schwierigkeiten:

\begin{enumerate}
  \item[a)] Schätzung des Fehlers
  \item[b)] Wahl der Zerlegung des Intervalls 
\end{enumerate}

\begin{nothing}[Zerlegung des Intervalls]
Für ein Teilintervall $[x_j, x_{j+1}]$ von $[a,b]$ lassen sich 
$$res[x_j, x_{j+1}] := h_{j+1} \sum_{i=1}^s b_i f(x_j + c_i h_{j+1})$$
und
$$ resabs[x_j, x_{j+1}] := h_{j+1} \sum_{i=1}^s \vert b_i f(x_j + c_i h_{j+1}) \vert$$
berechnen.\\
Angenommen wir können eine Schätzung des Fehlers $err[x_, x_{j+1}]$ berechnen mit
$$err[x_, x_{j+1}] \approx res[x_, x_{j+1}] - \int_{x_j}^{x_{j+1}} f(x) dx,$$
dann bietet sich folgendes Verfahren zur Konstruktion einer Zerlegung an:
\begin{enumerate}
  \item Berechne $res[a,b]$, $resabs[a,b]$ und $err[a, b]$. \\
    Falls 
    $$\vert err[a,b] \vert \leq Tol \thinspace resabs[a,b]$$
    Gebe $res[a,b]$ zurück.\\
    Ansonsten:
    
  \item Zerlege $[a,b]$ in 
    $$I_0 = \left[a,\frac{b-a}{2}\right]$$
    und
    $$I_1 = \left[ \frac{b-a}{2}, b \right]$$
    und berechne 
    \begin{description}
      \item $res \thinspace I_0$, $resabs \thinspace I_0$, $err \thinspace I_0$ und
      \item $res \thinspace I_1$, $resabs \thinspace I_1$, $err \thinspace I_1$ 
    \end{description}
    n = 2.
    
  \item Falls 
    $$ \sum_{j=0}^{n-1} \vert err \thinspace I_j \vert \leq Tol \thinspace \sum_{j=0}^{n-1} resabs \thinspace I_j$$
    Gebe
    $$ \sum_{j=0}^{n-1} res \thinspace I_j$$
    zurück. Ansonsten: \\
    Unterteile das Intervall $I_k = [a_k, b_k]$, in dem der Fehler maximal ist in zwei Teilintervalle 
    $$I_l = \left[a_k,\frac{b_k-a_k}{2}\right]$$
    und
    $$I_m = \left[ \frac{b_k-a_k}{2}, b_k \right]$$
    und berechne:
    \begin{description}
      \item $res \thinspace I_l$, $resabs \thinspace I_l$, $err \thinspace I_l$ und
      \item $res \thinspace I_m$, $resabs \thinspace I_m$, $err \thinspace I_m$
    \end{description}
    $n =n+1$ \\
    Gehe zu 3)
\end{enumerate}
\end{nothing}

\begin{nothing}[Schätzung des Fehlers]
\underline{Ziel:} Berechne Approximation an 
$$\int_{x_j}^{x_{j+1}} f(x)dx - h_{j+1} \sum_{i=1}^s b_i f(x_j + h_{j+1} c_i)$$
ohne zusätzliche Funktionsauswertungen.\\
\underline{Idee:} Konstruiere eingebettete QF, d.h. QF zu den selben Knoten $c_i$ mit Gewichten $\hat{b}_i$ und Ordnung $\hat{p} < p$. \\
\underline{Bemerkung:} Falls $p=2s$ ist, so gilt $\hat{p} \leq s-1$ (wäre $\hat{p} \geq s$, so wäre nach (2.8) $\hat{b}_i = b_i$).\\
Eine Approximation des Fehlers für die eingebettete QF ist durch 
\begin{align*}
\text{diff} \thinspace [x_j, x_{j+1}] &= h_{j+1} \sum_{i=1}^s b_i f(x_j + c_i h_{j+1}) - h_{j+1} \sum_{i=1}^s \hat{b}_i f(x_j + c_i h_{j+1})&\\
&= h_{j+1} \sum_{i=1}^s (b_i - \hat{b}_i) f(x_j+c_i h_{j+1})
\end{align*}
gegeben. Es gilt
\begin{align*}
\text{diff} \thinspace [x_j, x_{j+1}] &= h_{j+1} \sum_{i=1}^s b_i f(x_j + c_i h_{j+1}) - \int_{x_j}^{x_{j+1}} f(x)dx &\\
& \quad - \left( h_{j+1} \sum_{i=1}^s \hat{b}_i f(x_j + c_i h_{j+1}) - \int_{x_j}^{x_{j+1}} f(x)dx \right)&\\
&= \text{Fehler der QF } (b_i, c_i)_{i=1}^s - \text{Fehler der QF } (\hat{b}_i, c_i)_{i=1}^s &\\
&= C_1 h_{j+1}^{p+1} + C_2h_{j+1}^{\hat{p}+1}
\end{align*}
Falls $h_{j+1}$ klein ist, ist $C_1 h_{j+1}^{p+1} << C_2h_{j+1}^{\hat{p}+1}$.\\
Drei Möglichkeiten den Fehler zu schätzen:
\begin{enumerate}
  \item[I)] $\text{err }[x_j, x_{j+1}] \approx \text{diff }[x_j, x_{j+1}]$. Sehr pessimistisch
  \item[II)] $\text{err }[x_j, x_{j+1}] \approx (\text{diff }[x_j, x_{j+1}])^2$, falls $p=2s$ und $\hat{p} = s-1$. Wenig verlässlich
  \item[III)] Verwende dritte eingebettete QF 
    \begin{description}
      \item $(\hat{\hat{b}}_i, c_i)$ der Ordnung 6 
      \item zu $(b_i, c_i)$ der Ordnung $30 = 2s$, $s=15$
      \item und $(\hat{b}_i, c_i)$ der Ordnung 14
      \item $\hat{\text{diff}} = h_{j+1} \sum_{i=1}^s (b_i - \hat{\hat{b}}_i) f(x_{j} + c_i h_{j+1}) \approx C_3 h^7$
      \item 
        \begin{align*}
        \text{err }[x_j, x_{j+1}] &= \text{diff }[x_j, x_{j+1}] \left( \frac{\text{diff}}{\hat{\text{diff}}} \right) ^2&\\
            &= C_2 \frac{C_2^2}{C_3^2} h_{j+1}^{15} \left( \frac{h_{j+1}^{15}}{h_{j+1}^7} \right) ^2 = C h_{j+1}^{31}
        \end{align*}
    \end{description}
\end{enumerate}
\end{nothing}
 %Ein adaptives Programm
\subsection{Gauß- und Loballo Quadraturformeln}

\underline{Ziel:} Konstruktion einer s-stufigen QF der Ordnung $p=2s$.\\
Für $M(t) = CP_s(2t-1)$, wobei $P_s$ das Legendrepolynom vom Grad s ist (siehe (5.5)), $C \in \mathbb{R}$, erhalten wir mit (5.4) und (4.1):

\begin{theorem}
Für jedes $s \in \mathbb{N}$ gibt es eine eindeutige QF der Ordnung $p=2s$, die sogenannte Gauß-QF. Ihre Knoten sind die Wurzeln von $P_s(2t-1)$, ihre Gewichte sind durch (2.8) gegeben. 
\end{theorem}
\underline{Beispiele:} \\
\begin{tabular}{ll}
 
$s=1$ & Mittelpunktsregel \\

$s=2$ & $c_{1,2} = \frac{1}{2} \mp \frac{\sqrt{3}}{6}$, $b_1=\frac{1}{2} = b_2$ \\

$s=3$ & (4.3) 2) \\

\end{tabular}

\begin{nothing}[Bezeichnung der Knoten der Gauß-QF]
Details: Siehe Homepage (Übungsaufgabe). \\
\underline{Idee:} Die Wurzeln der Polynome, die durch Rekursion (5.3) erzeugt werden, sind die Eigenwerte einer symmetrischen Tridiagonalmatrix (Matrix: Siehe Homepage).\\
In Numerik II lernen Sie Verfahren kennen, um die Eigenwerte zu berechnen.
\end{nothing}

\begin{nothing}[Lobatto Quadraturformeln]
Ein Vorteil der Simpsonquadraturformel war, dass $c_1=0$ und $c_3=1$ gilt. Damit muss man den Integranten in $x_j$ nur einmal auswerten. Zur Konstruktion einer s-stufigen QF der Ordnung $p=2s-2$ mit $c_1=0$ und $c_s=1$ setzt man 
$$M(t) = P_s(2t-1) - P_{s-2}(2t-1)$$
Da die Legendre-Polynome folgende Rekursion erfüllen
$$P_0(x)=1 \quad P_1(x) = x $$
$$ (n+1)P_{n+1}(x) = (2n+1)xP_n(x) - nP_{n-1}(x)$$
ist 
$$ P_s(1) = 1 \quad \text{und} \quad P_s(-1) = (-1)^s$$
und damit 
$$M(0) = 0 = M(1)$$
Die restlichen Nullstellen (oder Wurzeln) von $M(t)$ sind reell, einfach und liegen in (0,1), wie man analog zu (5.4) zeigt.\\
Damit gilt:
\begin{description}
  \item[\textbf{Satz}]
    Für $s \in \mathbb{N}$, $s \geq 2$ gibt es eine eindeutige s-stufige QF der Ordnung $2s-2$ mit $c_1=0$ und $c_s=1$
\end{description}
\end{nothing} %Gauß- und Loballo Quadraturformeln

\section{Interpolation und Approximation}

\begin{description}
  \item[Problemstellung A]
    Zu gegebenen $(x_0, y_0), ...,(x_n, y_n)$ berechne Polynom $p$ vom Grad $\leq n$ mit $$p(x_j) = y_j, \quad j=0,...,n$$
  
  \item[Problemstellung B]
    $f:[a,b] \rightarrow \mathbb{R}$ gegeben. Finde einfach auszuwertende Funktion $p: [a,b] \rightarrow \mathbb{R}$, etwa ein Polynom, stückweises Polynom, rationale Funktion, sodass $f-p$ klein ist.
    \begin{enumerate}
      \item[i)] $f(x)=p(x)$ für endlich viele vorgegebene Punkte $x$
      \item[ii)] $\int_a^b (f(x)-p(x))^2 dx$ soll minimal sein.
      \item[iii)] $\max_{x \in [a,b]} \vert f(x) -p(x) \vert$ soll minimal sein.
    \end{enumerate}
\end{description}

\subsection{Newtonsche Interpolationsformel}

\begin{example}
\begin{description}\item \end{description}
\begin{description}
  \item n=1: \\
    $(x_0, y_0),(x_1,y_1)$, $p \in \mathcal{P}_1$ das beide Punkte verbindet.\\
    $$p(x) = y_0 + (x-x_0) \frac{y_1-y_0}{x_1-x_0}$$
  \item n=2: \\
    $(x_0, y_0),(x_1,y_1),(x_2,y_2)$ \\
    $$p(x) = y_0 + (x-x_0) \frac{y_1-y_0}{x_1-x_0} + a(x-x_0)(x-x_1)$$
    Bestimme $a$ so, dass $p(x_2) = y_2$
    \begin{align*}
    y_2 \overset{!}{=} y_0 + (x-x_0) \frac{y_1-y_0}{x_1-x_0} + a(x-x_0)(x-x_1)\\
    a(x_2-x_0)(x_2-x_1) = y_2 - y_0 - (x_2-x_1) \frac{y_1-y_0}{x_1-x_0} - y_1 + y_0 \\
    \Rightarrow a = \frac{1}{x_2-x_0} \left( \frac{y_2-y_1}{x_2-x_1} - \frac{y_1-y_0}{x_1-x_0} \right) 
     \end{align*}
\end{description}
\end{example}

\begin{definition}[dividierte Differenzen]
Für $(x_0,y_0), (x_1, y_1), ..., (x_n, y_n)$ mit paarweise verschiedenen Stützstellen $x_j$ definieren wir
$$ y[x_j] := y_j \quad \left( = \delta^0 y[x_j] \right) $$ 
$$ \delta y[x_j, x_{j+1}] := \frac{y_{j+1} - y_j}{x_{j+1}-x_j} = \frac{\delta^0 y[x_{j+1}]-\delta^0 y[x_{j}]}{x_{j+1} - x_j}$$
$$ \delta ^2 y[x_j, x_{j+1}, x_{j+2}] := \frac{\delta y[x_{j+1}, x_{j+2}]-\delta y[x_{j}, x_{j+1}]}{x_{j+2} - x_j}$$
$$ \delta ^k y[x_j, x_{j+1},..., x_{j+k}] := \frac{1}{x_{j+k}-x_j} \left( \delta^{k-1} y[x_{j+1}, ..., x_{j+k}] - \delta^{k-1} y[x_j, ..., x_{j+k-1}] \right)$$
\end{definition}

\end{document}