\subsection{Orthogonalpolynome}
Bedingung $2.$ in Satz $(4.1)$ 
$$ \forall h \in \mathcal{P}_{m-1}: \int_0^1 M(t)h(t) = 0$$
kann als Orthogonalitätsbedingung bzgl. eines Skalarprodukts $\langle f, g\rangle = \int_0^1 f(t)g(t)dt$ auf dem Vektorraum $L^2([0,1])$ oder $C([0,1])$ aufgefasst werden. \\
\underline{Erinnerung:}
$$\mathcal{P}_s := \left\{ \sum_{j=0}^s \alpha_j X^j, \alpha_j \in \mathbb{R} \right\}$$ 
ist ein $\mathbb{R}$-VR mit $dim(\mathcal{P}_s) = s+1$ und Basis $\left\{ 1, X, X^2, ..., X^s \right\}$\\ \\
$\langle\cdot,\cdot\rangle : C([0,1]) \times C([0,1]) \rightarrow \mathbb{R}, (f, g) \mapsto \int_0^1 f(t)g(t)dt$ ist 
\begin{enumerate}
  \item symmetrisch $ \langle f, g\rangle = \langle g, f\rangle$
  \item linear $\langle \alpha f + g, h\rangle = \alpha \langle f, h\rangle + \langle g, h\rangle$
  \item positiv definit $\langle f, f\rangle \geq 0 $ und $ \langle f, f\rangle = 0 \Rightarrow f = 0$
\end{enumerate}
Wie in der linearen Algebra definieren wir $f$ steht senkrecht auf $g$: $f \perp g \Leftrightarrow \langle f, g\rangle = 0$

\begin{theorem}
QF hat die Ordnung $s+m \Leftrightarrow $ M ist orthogonal auf allen Polynome in $\mathcal{P}_{m-1}$
\end{theorem}

\begin{definition}
Für eine Gewichtsfunktion $\omega : (a, b) \rightarrow \mathbb{R}$ mit 
\begin{enumerate}
  \item $\omega$ stetig
  \item $\forall x\in(a,b): \omega(x) > 0 $
  \item $\forall k \in \mathbb{N}: \int_a^b \omega(x) \vert x \vert^k dx < \infty$
\end{enumerate}
definieren wir auf den Vektorraum 
$$ V = \left\{ f: [a,b] \rightarrow \mathbb{R}: f \medspace stetig \medspace und \int_a^b f(x)^2 \omega(x) dx < \infty \right\} $$
das gewichtete Skalarprodukt
$$ \langle f, g \rangle_\omega := \int_a^b \omega(x)f(x)g(x)dx$$
für $f, g \in V$. \\
Zudem definiere:
$$f \perp_\omega g :\Leftrightarrow \langle f, g, \rangle_\omega = 0$$
\end{definition}

\begin{theorem}
Es existiert eine eindeutige Folge von Polynomen $p_0, p_1, ...$ mit 
\begin{enumerate}
  \item $deg(p_k) = k$
  \item $\forall q \in \mathcal{P}_{k-1}:p_k \perp q$ für $k \geq 1$
  \item $p_k(x) = x^k + r$ mit $deg(r) \leq  k-1$ "Normierung"
\end{enumerate}
Diese Polynome lassen sich rekursiv berechnen durch
\begin{flalign*}
&p_0(x) := 1\\
&p_{1}(x) := x \\
&p_{k+1}(x) := (x- \beta_{k+1}) p_k(x) - \gamma_{k+1}^2 p_{k-1}(x), \quad \text{für $k \geq 2$} \\
\intertext{Wobei $\beta$ und $\gamma$ definiert sind durch:}
&\beta_{k+1} := \frac{\langle xp_k, p_k \rangle}{\langle p_k, p_k \rangle} \\
&\gamma_{k+1}^2 := \frac{\langle p_k, p_k \rangle}{\langle p_{k-1}, p_{k-1} \rangle}
\end{flalign*}

\begin{proof}[Beweis]
(vgl. Gram-Schmidt Orthogonalisierung LinA) \\
Sei $p_0, ..., p_k$ bereits bekannt. Zur Konstruktion von $p_{k+1}$ setzen wir
$$p_{k+1}(x) = xp_k(x) + \sum_{j=0}^{k} \alpha_j p_j(x)$$
(damit ist 3. erfüllt) \\
Zur Bestimmung der $\alpha_j$:
\begin{enumerate}
  \item $ 0 = \langle p_{k+1}, p_k \rangle = \langle xp_k, p_k \rangle + \alpha_k \langle p_k, p_k \rangle + \sum_{j=0}^{k-1} \alpha_j \underbrace{\langle p_j, p_k \rangle}_{= 0}$\\
    $\Rightarrow \alpha_k = -\frac{\langle xp_k, p_k \rangle}{\langle p_k, p_k \rangle} =: -\beta_{k+1}$
  
  \item $$0 = \langle p_{k+1}, p_{k-1} \rangle = \langle xp_k, p_{k-1} \rangle + 0 + \alpha_{k-1} \langle p_{k-1}, p_{k-1} \rangle + 0$$ 
    $$= \langle p_{k}, xp_{k-1} \rangle + \alpha_{k-1} \langle p_{k-1}, p_{k-1} \rangle$$ 
    Aufgrund von 3. $\Rightarrow$ 
    $$xp_{k-1} = p_k + r$$ mit $deg(r) \leq k-1$
    $$\Rightarrow \langle p_{k}, xp_{k-1} \rangle = \langle p_{k}, p_{k} \rangle + \underbrace{\langle p_{k}, r \rangle}_{= 0}$$ 
    $$\Rightarrow \alpha_{k-1} = - \frac{\langle p_k, p_k \rangle}{\langle p_{k-1}, p_{k-1} \rangle} =: -\gamma_{k+1}^2$$
  
  \item Für $j \leq k-2$:
    $$ 0 = \langle p_{k+1}, p_{j} \rangle = \langle xp_{k}, p_{j} \rangle + \alpha_j \langle p_{j}, p_{j} \rangle$$
    $$ = \underbrace{\langle p_{k}, xp_{j} \rangle}_{= 0} + \alpha_j \underbrace{\langle p_{j}, p_{j} \rangle}_{\neq 0}$$
    $\langle p_{k}, xp_{j} \rangle = 0$ gilt, da $deg(xp_j) \leq k+1$ \\
    Insgesamt haben wir
    $$p_{k+1}(x) = xp_k(x) - \beta_{k+1}p_k(x) - \gamma_{k+1}^2 p_{k-1}(x)$$
\end{enumerate}
\end{proof}
\end{theorem}

\begin{comment*}
Für eine QF maximaler Ordnung müssen nach Satz (4.1) die Knoten $c_i$, $i=1, ...,s$ so gewählt werden, dass 
$$M(t) = \prod_{i=1}^s(t-c_i)$$
das Orthogonalpolynom vom Grad $s$ bezüglich des Skalarprodukts mit $\omega(x) \equiv 1$ auf $[0,1]$ ist. \\
    \underline{Frage:} Sind die Wurzeln (Nullstellen) der Orthogonalpolynome aus (5.3) reell? (Spoiler: Ja)
\end{comment*}

\begin{theorem}
Sei $p_k$ das Orthogonalpolynom wie in (5.3) definiert (bzgl. $\langle f, g \rangle = \int_a^b f(x)g(x)\omega(x)dx$). Alle Wurzeln von $p_k$ sind einfach und liegen im offenen Intervall $(a,b)$.

\begin{proof}[Beweis]
Seie $x_1, ..., x_r$ jene Wurzeln in $p_k$, die reell sind, in $(a, b)$ liegen und bei denen $p_k$ das Vorzeichen wechselt (Wurzeln mit ungerader Vielfachheit). \\
Klar ist: $r \leq k$. \\
Sei 
$$g(x) = \prod_{j=1}^r (x-x_j)$$ 
Dann ist 
$$ \langle p_k, g \rangle = \int_a^b \underbrace{p_k(x) \thinspace g(x)}_{\text{Wechselt das Vorzeichen in (a,b) nicht}} \thinspace \omega(x) \thinspace dx \neq 0$$
Andererseits ist $p_k$ orthogonal zu allen Polynomen vom Grad $\leq k-1$ \\
$\Rightarrow r = deg(g) \geq k$ \\
$\Rightarrow r=k$
\end{proof}
\end{theorem}

\begin{example}[Orthogonale Polynome]
\begin{tabular}{llll}
 
Bezeichnung & $(a,b)$ & $\omega(x)$ & Name\\
 
& & & \\

$P_k$ & $(-1,1)$ & $1$ & Legendrepolynome \\

$T_k$ & $(-1,1)$ & $(1-x^2)^{-1/2}$ & Tschebyscheff-Polynome \\

$P_k^{(\alpha, \beta)}$ & $(-1,1)$ & $(1-x)^{\alpha}(1-x)^{\beta}$ & Jacobi-Polynome $\alpha, \beta > -1$ \\

$L_k^{(\alpha)}$ & $(0, \infty)$ & $x^{\alpha} e^{-x}$ & Laguere-Polynome \\

$M_k$ & $(-\infty,\infty)$ & $e^{-x^2}$ & Harmitepolynome \\

\end{tabular}\\
\underline{Bemerkung:} Teilweise sind andere Normierungen üblich $P_k(1) = 1$, $T_k(x) = 2^{k-1} x^k + ...$, ...
\end{example}
